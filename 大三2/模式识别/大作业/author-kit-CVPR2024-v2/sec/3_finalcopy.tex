
\section{Applications of Scene Understanding}

Scene understanding is important in several practical applications, and the following are a few key application areas:

    	\textbf{Autonomous Driving}: In the field of autonomous driving, vehicles require real-time and accurate perception and understanding of the surrounding environment to make safe driving decisions. Scene understanding technologies assist autonomous driving systems in identifying objects such as vehicles, pedestrians, and traffic signs on the road, and understanding their relationships, thereby enhancing the safety and reliability of autonomous driving.

    	\textbf{Intelligent Surveillance}: Intelligent surveillance systems monitor specific areas using cameras, requiring real-time detection and recognition of anomalies such as intrusion and fighting behaviors. Scene understanding technologies improve the automation of surveillance systems, reduce manual intervention, and enhance monitoring efficiency and accuracy.

    	\textbf{Augmented Reality (AR)}: In augmented reality (AR) applications, scene understanding technologies are used to identify objects and environments in the real world and overlay virtual information on them. For example, in educational and entertainment applications, AR systems can recognize images in textbooks and display related 3D models or animations, enhancing user experience.

	\textbf{Robot Navigation}: Robots performing tasks in complex environments rely on scene understanding technologies to perceive and comprehend the surrounding environment for path planning and obstacle avoidance. For instance, service robots in home environments need to recognize and locate objects such as furniture and appliances to complete tasks like delivery and cleaning.

\section{Conclusion}
This review explores the downstream applications of open-vocabulary visual perception technologies in scene understanding, analyzing three pivotal papers: "OpenMask3D: Open-Vocabulary 3D Instance Segmentation," "From Pixels to Graphs: Open-Vocabulary Scene Graph Generation with Vision-Language Models," and "OVER-NAV: Elevating Iterative Vision-and-Language Navigation with Open-Vocabulary Detection and Structured Representation." These papers demonstrate the latest advancements in the field and their wide-ranging applications.

\textbf{Open-Vocabulary Scene Graph Generation}

The introduction of a scene graph generation framework from image-to-text in open-vocabulary settings has significantly enhanced both the accuracy of scene graph generation and the performance in downstream vision-language tasks, showcasing the robust capabilities of open-vocabulary systems in complex scene understanding.

\textbf{Open-Vocabulary 3D Instance Segmentation}

OpenMask3D proposes a two-stage method based on the CLIP model for open-vocabulary 3D instance segmentation. It achieves this through category-agnostic mask proposals and multi-view feature aggregation, overcoming the limitations of traditional closed-vocabulary methods and significantly improving system performance in handling novel and diverse objects.

\textbf{Open-Vocabulary Navigation}

OVER-NAV integrates large language models and open-vocabulary detection techniques to propose a new iterative vision-and-language navigation framework. By constructing the structured memory Omnigraph, this approach markedly enhances the navigation capabilities of agents in unknown environments. This method underscores the importance of multimodal information fusion and demonstrates its potential applications in navigation tasks.


In summary, open-vocabulary visual perception technologies exhibit tremendous potential and extensive applications in scene understanding. By leveraging advanced techniques such as vision-language models, open-vocabulary detection, and structured representations, these methods surpass the limitations of traditional closed-vocabulary systems, providing more flexible and intelligent solutions. Future research should continue exploring these technologies across various domains to further advance the development of visual perception and artificial intelligence.

